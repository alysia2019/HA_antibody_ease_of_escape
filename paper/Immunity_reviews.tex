\documentclass[11pt, oneside]{article}   	% use "amsart" instead of "article" for AMSLaTeX format
\usepackage{geometry}                		% See geometry.pdf to learn the layout options. There are lots.
\geometry{letterpaper}                   		% ... or a4paper or a5paper or ... 
%\geometry{landscape}                		% Activate for rotated page geometry
%\usepackage[parfill]{parskip}    		% Activate to begin paragraphs with an empty line rather than an indent
\usepackage{graphicx}				% Use pdf, png, jpg, or eps§ with pdflatex; use eps in DVI mode
								% TeX will automatically convert eps --> pdf in pdflatex		
\usepackage{amssymb}

\usepackage{color}

\usepackage[parfill]{parskip}

%SetFonts

%SetFonts


\title{Reviewers' comments from \textit{Immunity}}
\author{}
%\date{}							% Activate to display a given date or no date

\begin{document}
\maketitle

\section*{Editor Comments}

Dear Dr. Bloom,

Thank you for sending Immunity your manuscript titled "Quantifying the ease of viral escape from broad and narrow antibodies to influenza hemagglutinin" (IMMUNITY-D-17-01026). We have received the reports of the reviewers and we enclose the comments they have provided for transmission to the authors. Unfortunately, the recommendation of the referees is against publication in Immunity.

Jesse- I'm sorry for the length of the review process. Thank you for your patience. Unfortunately, it was felt that the manuscript does not provide a sufficient level of conceptual advance over work published by your lab and others, and there were also concerns with the physiological relevance of the findings and whether the conclusions are broadly relevant and generalizable. Unfortunately, the level of enthusiasm for the work was not sufficient for us to invite a revision, and we feel that it would be more efficient to seek another venue for the work. I know that this manuscript represents a substantial amount of careful work, and I?m sorry that I can?t provide you with a better outcome on this occasion. The manuscript may be a better fit for another Cell Press journal. I have provided links below that can be used should you wish to transfer the manuscript. Please let me know if you have any questions about transfers or if I can be of any further assistance. -Kavitha

\section*{Reviewer \#1: Summary}

Doud et al. used mutational antigenic profiling to systematically map the influenza HA resistant mutations against three broadly neutralizing antibodies - one targets the HA receptor-bind site (S139/1), and two target the HA stalk region (C179 and FI6v3). The authors then analyzed the data generated in this study along with the data generated in their previous study (Doud et al., PLoS Pathog 2017), which describes the mapping of influenza HA resistant mutations against three strain-specific neutralizing antibodies - H17-L7, H17-L10, and H17-L19. Here, the authors made a slight improvement in the analytical method that was originally described in their previous study (Doud et al., PLoS Pathog 2017). The improvement is based on the introduction of the experimentally determined parameter $\gamma$, which describes the fraction of viral library that survives the antibody selection, and is measured using RT-qPCR. Using this improved analytical method, the authors observed strong resistance mutations to S139/1 (broad), H17-L7 (strain-specific), H17-L10 (strain-specific), and H17-L19 (strain-specific), but not to C179 (broad) and FI6v3 (broad). Only weak resistant mutations, including previously unknown ones, were identified for C179 and FI6v3. Based on this result, Doud et al. claimed that antibody breadth is not necessarily an indicator of the difficulty of viral escape and stalk-binding antibodies are more difficult to escape. By pulling the data from their previous study (Doud and Bloom, Viruses 2016), the authors further showed that the lack of strong resistant mutations to stalk-binding antibodies (C179 and FI6v3) is not due to the lack of mutational tolerance.

While this paper is technically sound, the novelty and biological insights are limited. Despite the slight improvement, which allows results from mutational antigenic profiling to be normalized and compared across different antibodies, the methodology that Doud et al. employed is largely the same as they described in their previous study (Doud et al., PLoS Pathog 2017). Also, the mutant libraries used in this study have already been described in authors' previous studies (Doud and Bloom, Viruses 2016; Doud et al., PLoS Pathog 2017). The claim that antibody breadth is not correlated to the ease of viral escape is potentially interesting, but largely relies on the observation of resistant mutations to only one broadly neutralizing antibody S139/1. Therefore, the generality of this claim is unclear.

{\color{red} This is a good and fair summary.
We should make sure that we are not over-stating the generality of the claims.}

Specific comments:

1. Based on Figure 4 and 5, the number of residues that can give rise to resistant mutations to S139/1 seem to be less than to two of the strain-specific antibodies (H17-L10 and H17-L19), and maybe also to H17-L7. Does that suggest S139/1 is slightly more difficult to escape as compared to strain-specific antibodies?
{\color{red} We could comment briefly on the number of sites of escape for the different residues.}

2. Related to the concern above, although the methodology in this study normalized the antigenic profiling results from different antibodies, a quantitative comparison of the ease of escape among different antibodies was not performed. Is it possible to rank the ease of escape among different antibodies?
{\color{red} We could easily rank the antibodies by something like the median effect of the top 5 escape mutations. Mike had a figure that looked a bit like that.}

3. In Figure 6A, the fraction infectivity of K(-8)T at low concentration of FI6v3 is consistent higher that that of WT. Is there any explanation? On a related point the resistant mutation to FI6v3 at residue -8 is an interesting observation. Can the authors suggest possible mechanism?
{\color{red} I think the effect is so small that we aren't even sure that it is true, which is why we didn't really discuss this.. We could mention something about possibly modulating expression level...}

4. In Figure 6A, V135T seems to show some resistance at FI6v3 at level that is similar to K(-8)T, although the authors claim that V135T does not affect the neutralization activity of FI6v3. Can the authors quantify the inhibition effect (e.g. IC50) in this validation experiment to facilitate interpretation of the results?
{\color{red} I'm not sure if we gain anything by trying to do statistics on this, since the data is all shown and the effect is clearly tiny. 
Might be best just to remain very circumspect about K(-8)T?}

5. In the Figure 2B and in the methods section, the authors state that S139/1 has not been tested against H9. However, S139/1 has been shown to binding to H9 by ELISA but without any neutralization activity (Yoshida et al., PLoS Pathog 2009).
{\color{red} We should add this fact.}

\section*{Reviewer \#2} 
This is a manuscript examining the escape of influenza to broad and narrow antibodies in order to quantify escape. It is a well-written paper and the authors are to be commended for their encyclopedic referencing of classic literature. I appreciate the desire of the authors to perform an "apples-to-apples" comparison, and there is much to like about this paper, but unfortunately, this analysis has flaws that preclude me recommending it for publication.

The manuscript focuses on neutralization as the primary mechanism of pressure on influenza and uses neutralization measurements, antibody selection of viruses, and a model to interpret the results. In particular, the authors compare an antibody that binds near the receptor binding site (S139/1), two stem-directed antibodies (FI6v3 and C179), and a series of strain-specific antibodies (H17 series).

Specific concerns

1. Although mAb S139/1 is called a receptor binding site (RBS) antibody, the epitope is actually adjacent to the RBS. The abstract of the Yoshida paper called it "a novel conformational epitope adjacent to the receptor-binding domain of HA." It is not clear what a true RBS antibody (eg, 5J8, CH65) would do in this model.
{\color{red} We can certainly say that it binds \emph{near} the RBS rather than in the RBS.
I don't think it is worth adding additional antibodies at this time.}

2. Stem epitopes and the RBS have different degrees of exposure on properly formed, infectious influenza virions. Although a number of publications have shown that stem antibodies can access the epitope on infectious virions, the data shows that 1-3 orders of magnitude more antibody is needed to neutralize infectious virions compared with pseudovirions (where HA packing is less dense) [see Joyce MG et al. Cell 166(3): 609-623 (2016) and Corti D et al. Science 333(6044): 850-856 (2011)]. This raises the question of whether a model dependent on neutralization of infectious virions is relevant to assess in vivo pressure on the HA stem epitope. Looking at the Doud and Bloom 2016 paper, it is not clear whether the virions produced in the helper virus system would be packed tightly (as infectious influenza is) or loosely (like in the pseudovirus assay).
{\color{red}
This is one that we could address if given a chance.
We are using full-replication competent viruses that should have physiological HA density. 
We can add a line that mentions this.
}

3. In vivo protection afforded by stem-directed antibodies has been convincingly shown to be Fc-FcR mediated [see DiLillo DJ et al. J Clin Invest 126(2): 605-610 (2016) and DiLillo DJ et al. Nat Med 20(2): 143-151 (2014)], making the relevance of a finding based on neutralization (whether tightly or loosely packed) unclear.
{\color{red}
We can mention something about FcR.
However, I would guess that mutations that escape neutralization probably also escape any in vivo activity.
}

4. The amount of antibody used is wildly different among the antibodies tested and it is not clear why some antibodies (eg, S139/1) needed very high concentrations in the assay while others (eg, C179) required two orders of magnitude less. This is particularly relevant given the comment above about epitopic exposure on virions.
{\color{red}
We should emphasize that we are choosing concentrations where about the same \emph{fraction} (between 0.01 and 0.001) of the library escapes the antibody.
The neutralization curves are for wildtype library, and these fractions are computed on the mutant libraries.
}

5. The neutralization curves (Fig. 3) that show all antibodies had complete or near-complete neutralization at 1 $\mu$g/mL. Since the point is to have an ``apples-to-apples'' comparison, it would seem that comparing the ability of antibodies to select viruses at similar concentrations, and especially at concentrations likely to be present in a physiologic condition, would be the relevant assay.
{\color{red}
See answer above.
}


\section*{Reviewer \#3} 
Boud et al., used a previously characterized HA single AA mutant library and a previously developed method for quantifying viral survival to quantify the viral escape from three broad neutralizing antibodies. While the work is interesting, there are several problems.

1. Their previously published HA sequencing method uses a molecular barcode, which helps to correct the sequencing and PCR errors. However, they did not explain why they still need to use a ratio between antibody selected and mock treated to observe the true mutant.
{\color{red} 
This is a dumb comment. 
I'm not sure if it is even worth trying to clarify in the manuscript since I think it is obvious to anyone who thinks about it.
}

2. Some of the mutants may have a competitive edge over other mutants regardless of antibodies, while others, the inhibition of other competitive mutants by antibodies may give them some advantage. As their data in Figure 7 revealed, sometime the selected mutation is rather disfavored with respect to viral growth, which would affect the growth of these survivors if they are in the original viral pool without antibody selection. Thus, the use of ratio makes it impossible to distinguish these two scenarios. Also, would these selected mutants that are disfavor the viral growth develop additional mutation to favor the viral growth in long-term?
{\color{red}
This also seems like a dumb comment.
I think we make fairly clear that the goal is to identify mutations that mediate escape, not to quantify their effect on viral growth.
I guess maybe the term ``ease of escape'' is confusing in this sense?
We are pretty clear in the Discussion that we are only looking at single mutations, and that there could be compensatory changes -- we spend a whole paragraph on this.
}

3. Also, how did authors make sure there is no drop out in the library at the beginning of the assay?
{\color{red}
These plots are in the notebook. 
I suppose we could move some of them to be supporting figures -- or just refer to the notebook more clearly?
}

4. Full HA sequences diversity vs abundance curve need to be shown in order for readers to fully understand the dynamic changes of viral species before and after antibody addition.
{\color{red}
See comment above.
}

5. Part of the data, the narrow antibody selection, are from their previous publication with a similar conclusion, although the mutation patterns have some differences. The new analysis applied a background correction so that they can compare data obtained from different experiments. So the question is how reproducible is their method? Do they have some kind of metrics for false negative and false positive based on their criterial?
{\color{red}
This also sort of a dumb comment.
It's actually a good sign that we get essentially the same results using two slightly different methods.
The point of data analyses is that the conclusions should be robust to different reasonable analysis methods, not that the results should be identical.
I don't think we can really develop a metric for false positive / negative rates.
}

6. Why they choose to use P=5 for this study while 10 in the previous one? How that affects the results?
{\color{red}
The pseudocount choice is inherently arbitrary.
See the comment above.
}

7. Figure 6 lacks statistic tests.
{\color{red}
I still think it is better to show the actual data than run some semi-bogus statistical test.
But we could add some sort of test statistic I suppose...}

8. For the last result section, I do not think they have enough experimental evidence to favor one hypothesis over the other without testing dependent mutations.
{\color{red}
I don't think we draw any overly strong conclusions here, but I suppose we could be a bit more circumspect.
}


\end{document}  